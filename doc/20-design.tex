\chapter{Конструкторский раздел}

\section{Проектирование загружаемого модуля ядра}

При загрузке модуль ядра не получает никаких параметров: для обеспечения большей безопасности серийный номер USB-ключа зашит в коде.

Очевидно, модуль ядра будет загружаться при запуске операционной системы.
При этом проверку серийного номера USB-устройства необходимо произвести до аутентификации с помощью пароля.

В Linux программа, которая управляет командной строкой и соответственно подключённым терминалом, называется \code{agetty}.
Её цель~— защитить систему от несанкционированного доступа.
Большинство времени процесс \code{agetty} находится в состоянии прерываемого сна (\code{TASK\_INTERRUPTIBLE}) в ожидании ввода.
Одним из способов приостановления возможности аутентификации пользователя с помощью пароля является переключение процесса \code{agetty} в состояние непрерываемого сна (\code{TASK\_UNINTERRUPITBLE}).

При загрузке модуля происходит регистрация USB-драйвера и запуск потока блокировки \code{agetty}.
В обработчике подключения USB-устройств драйвера выполняется проверка серийного номера и, в случае совпадения, остановка потока блокировки \code{agetty}.
По истечении 30 секунд при отсутствии подключённого USB-ключа процесс блокировки \code{agetty} завершает работу операционной системы.

\subsection{Алгоритм установки состояния процесса}

Оформим отдельной подпрограммой алгоритм установки состояния процесса (рис. \ref{img:uba-set-task-state}).
Функция возвращает \code{true}, если процесс с заданным именем был найден и успешно переведён в требуемое состояние.
Иначе~— \code{false}.

\img{width=\linewidth}{uba-set-task-state}{Схема алгоритма установки состояния процесса}

\subsection{Алгоритм функции потока блокировки}

На рисунке \ref{img:uba-suspend-agetty-process} представлена схема алгоритма функции потока блокировки процесса \code{agetty}.

Сначала процесс \code{agetty} переключается в состояние непрерываемого сна, затем производится обратный отсчёт до завершения работы системы, остановить который может лишь сигнал об остановке потока.

\img{width=0.90\linewidth}{uba-suspend-agetty-process}{Схема алгоритма функции потока блокировки процесса \code{agetty}}

\subsection{Алгоритм проверки подлинности ключа}

На рисунке \ref{img:uba-probe} представлена схема алгоритма проверки подлинности USB-ключа.

После успешной аутентификации дальнейшие проверки подлинности не производятся, процесс \code{agetty} переводится обратно в состояние прерываемого сна.

\img{width=0.98\linewidth}{uba-probe}{Схема алгоритма проверки подлинности USB-ключа}

\section*{Вывод}

В результате проектирования разработаны алгоритмы проверки подлинности и функции потока блокировки, что позволяет перейти к реализации загружаемого модуля в программном коде.
