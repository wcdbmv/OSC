\chapter{Конструкторский раздел}

В конструкторском разделе необходимо продемонстрировать место разрабатываемого ПО в системе и в ядре, в частности. Определяется тип программного обеспечения, например, драйвер и приложение, взаимодействующее с драйвером, или драйвер и сервис, или для Linux драйвер и демон, и описывается взаимодействие модулей разработанного ПО через подсистему ввода/вывода, например, на основе событийной модели.

Описать структуру разрабатываемого драйвера, основные структуры ядра, описывающие драйвер и функции драйвера (точки входа).

Описывается реализуемая драйвером функция или функции и приводится схема алгоритма или схемы алгоритмов их работы по ГОСТу.

Допустимо при необходимости представления структуры классов сделать обоснованный выбор объектно-ориентированного подхода и средств разработки (например, для ОС Windows, выбор библиотеки NuMega Driver Studio, которая является объектно-ориентированной надстройкой над «чистым» DDK).

Например, для драйверов блочных устройств Unix рассматриваются особенности обмена с блочным устройством и структура buf.

Для WDF описываются объекты, которые представляют абстракции компонентов драйверов, и методы, свойства, события, посредством которых драйверы взаимодействуют с объектами WDF.

В конструкторском разделе следует:
\begin{itemize}
	\item обосновать какие части драйвера могут находиться в перемещаемой памяти, а какие – должны оставаться резидентными;
	\item описать процесс обработки пакетов запроса ввода/вывода (I/O Request Packet или IRP) и выбрать сценарий обработки IRP [Oni]:
	\item обосновать необходимость применения отложенных процедурных вызовов DPC (для Windows):
	\item обосновать метод буферизации, например:

	в WDM-драйверах запрос может иметь или буфер ( для буферизованного ввода/вывода ), или список MDL ( для прямого ввода/вывода ); в модели WDF можно абстрагироваться от типа запроса ( буферизованного или прямого ) [орвик]; для драйверов Unix – выбор метода
	буферизации: используя прерывания, с помощью функции poll(), с помощью специальных
	структур данных clist.

	\item если в драйвере создается дополнительный поток, обосновать необходимость его
	создания и выбор средств взаимоисключения;

	\item при необходимости использовать события WMI (Windows Management Instrumentation)
	для оповещения потребителей об интересных или экстренных событиях;
\end{itemize}

\section*{Вывод}