\chapter{Технологический раздел}

\textit{Обоснованно выбирается язык программирования и среда программирования (следует определить основные особенности языка, среды и специализированных библиотек и на этом основании сделать выбор языка программирования, средства или библиотеки).}

\textit{Для драйверов, демонов или сервисов подробно описывается среда разработки; приводятся выбранные структуры данных и разработанные функции, реализующие поставленную задачу.
Для патчей описать особенности реализации и перекомпиляции ядра ОС.}

\textit{Обосновывается выбор интерфейса пользователя и приводится его описание; описываются действия по установке программного обеспечения.}

\textit{Например для Windows, установка драйвера в систему с помощью .INF файла с указанием необходимого набора файлов или с помощью установочного приложения; при необходимости разрабатываются демонстрационные программы; при необходимости разрабатываются отладочные программы и тестирующие программы или описываются использованные средства отладки и тестирования, например, тестирование средствами DDK помощью стандартной тестирующей утилиты экспериментальная проверка программного обеспечения ( тестирование ), анализ результатов экспериментальной проверки разработанного программного обеспечения или полученных результатов исследования параметров системы; результаты исследования следует оформлять в виде графиков и/или диаграмм, таблиц.}

\textit{Для Linux make файл.}

\section{Выбор языка программирования}

Перечислим основные особенности, которыми должен обладать язык программирования для разработки загружаемого модуля ядра Linux:
\begin{itemize}
	\item язык высокого уровня;
	\item широко используется для разработки загружаемых модулей ядра Linux;
	\item компилируемый;
	\item модульный;
	\item быстрый;
	\item компактный.
\end{itemize}

Всеми перечисленными особенностями обладает только язык C, поэтому именно он и выбран в качестве языка программирования для разработки загружаемого модуля ядра Linux.
Такой выбор не удивителен — язык C первоначально создавался как язык системного программирования.
В исходном коде ядра Linux доля исполняемых файлов на языке C составляет приблизительно 97 \% \cite{github-linux}.

%В качестве языка программирования для реализации загружаемого модуля ядра был выбран язык C, так как ядро операционной системы Linux написано в основном на нём.
%% https://www.kernel.org/doc/html/latest/process/howto.html#introduction
%C является языком высокого уровня, первоначально создавался как язык системного программирования.
%% https://www.codingunit.com/the-history-of-the-c-language
%язык С создавался для того, чтобы переписать ядро на языке более высокого уровня, выполняющем те же задачи с использованием меньшего количества строк кода.
%%Основные особенности языка C:
%\begin{itemize}
%	\item компилируемый,
%	\item со слабой статической типизацией,
%	\item общего назначения,
%	\item модульный,
%	\item процедурный,
%	\item малое количество ключевых слов,
%	\item использует препроцессор,
%	\item предоставляет низкоуровневый доступ к памяти через использование указателей.
%\end{itemize}
%% https://ru.wikipedia.org/wiki/%D0%A1%D0%B8_(%D1%8F%D0%B7%D1%8B%D0%BA_%D0%BF%D1%80%D0%BE%D0%B3%D1%80%D0%B0%D0%BC%D0%BC%D0%B8%D1%80%D0%BE%D0%B2%D0%B0%D0%BD%D0%B8%D1%8F)#%D0%9E%D0%B1%D1%89%D0%B8%D0%B5_%D1%81%D0%B2%D0%B5%D0%B4%D0%B5%D0%BD%D0%B8%D1%8F
%% http://progopedia.ru/language/c/

\section{Выбор среды программирования}

Теперь перечислим основные особенности среды программирования для разработки загружаемого модуля ядра Linux.
\begin{itemize}
	\item текстовый интерфейс;
	\item компактность;
	\item быстродействие;
	\item удобство использования
	\item широкое распространение среди разработчиков ядра Linux;
\end{itemize}

Интегрированные среды разработки с графическим интерфейсом не обладают всеми перечисленными особенностями.
Поэтому среда программирования выбрана покомпонентно:
\begin{itemize}
	\item текстовый редактор — vim,
	\item компилятор — gcc,
	\item средство автоматизации сборки — make.
\end{itemize}

\section*{Вывод}
