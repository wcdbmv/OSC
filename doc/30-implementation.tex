\chapter{Технологический раздел}

\section{Выбор языка программирования}

Перечислим основные особенности, которыми должен обладать язык программирования для разработки загружаемого модуля ядра операционной системы Linux:
\begin{itemize}
	\item язык высокого уровня;
	\item широко используется для разработки загружаемых модулей ядра Linux;
	\item компилируемый;
	\item модульный;
	\item быстрый;
	\item компактный.
\end{itemize}

Всеми перечисленными особенностями обладает только язык C, поэтому именно он и выбран в качестве языка программирования для разработки загружаемого модуля ядра Linux.
Такой выбор не удивителен — язык C первоначально создавался как язык системного программирования.
В исходном коде ядра Linux доля исполняемых файлов на языке C составляет приблизительно 97 \% \cite{github-linux}.

\section{Выбор среды программирования}

Теперь перечислим основные особенности, которыми должна обладать среда программирования для разработки загружаемого модуля ядра ОС Linux.
\begin{itemize}
	\item текстовый интерфейс;
	\item компактность;
	\item быстродействие;
	\item удобство использования
	\item широкое распространение среди разработчиков ядра Linux;
\end{itemize}

Интегрированные среды разработки с графическим интерфейсом не обладают всеми перечисленными особенностями.
Поэтому среда программирования выбрана покомпонентно:
\begin{itemize}
	\item текстовый редактор — vim,
	\item компилятор — gcc,
	\item средство автоматизации сборки — make.
\end{itemize}

\section{Реализация загружаемого модуля ядра}

В листинге \ref{lst:usb-boot-authentication} (приложение А, стр. \pageref{chp:attachment-a}) представлен исходный код загружаемого модуля ядра.

Предварительно необходимо задать собственный серийный номер USB-ключа \code{UBA\_SERIAL}, узнать который можно, например, просмотром системного журнала командой \code{\$ dmesg} после подключения.

\section{Действия по установке ПО}

В листинге \ref{lst:makefile} (приложение B, стр. \pageref{chp:attachment-b}) представлен Makefile загружаемого модуля ядра.

Действия по установке ПО показаны в листинге \ref{lst:install}.

\begin{lstlisting}[
	basicstyle=\small\ttfamily,
	numberstyle=\footnotesize\ttfamily\color{gray},
	caption={Установка\label{lst:install}},
	gobble=8,
	language=Bash
]
	$ make
	# make enable_at_boot
	# make boot_in_console_mode
	# make enable_printing_kernel_journal_on_tty
\end{lstlisting}

Рассмотрим детальнее каждый шаг.

\begin{itemize}
	\item Цель по умолчанию (\code{make} без параметров) производит сборку загружаемого модуля ядра.
	\item Цель \code{enable\_at\_boot} помещает название разработанного модуля в \code{/etc/modules}, тем самым обеспечивается автоматическая загрузка модуля ядра при запуске операционной системы.
	\item Цель \code{boot\_in\_console\_mode} изменяет конфигурационные файлы загрузчика \code{grub2} и подсистемы инициализации и управления службами \code{systemd} для запуска операционной системы в консольном режиме.
	\item Цель \code{enable\_printing\_kernel\_journal\_on\_tty} создаёт \code{rsyslog}-конфиг, позволяющий увидеть сообщения разработанного модуля ядра в терминале.
\end{itemize}

Выполнение последней цели является необязательным для функционирования программы, однако пропуск этого шага уместен лишь в том случае, если есть необходимость в сокрытии от пользователя информации, объясняющей порядка аутентификации.

К каждой перечисленной цели в Makefile имеется противоположная, действия по удалению ПО показаны в листинге \ref{lst:uninstall}.

\begin{lstlisting}[
	basicstyle=\small\ttfamily,
	numberstyle=\footnotesize\ttfamily\color{gray},
	caption={Удаление\label{lst:uninstall}},
	gobble=8,
	language=Bash
]
	# make disable_printing_kernel_journal_on_tty
	# make boot_in_graphical_mode
	# make disable_at_boot
	$ make clean
\end{lstlisting}

При необходимости использования графического интерфейса пользователя после успешного прохождения двухфакторной аутентификации следует ввести

\begin{verbatim}
	$ systemctl isolate graphical.target
\end{verbatim}

\section*{Вывод}

В результате разработки загружаемого модуля ядра получено программное обеспечение, обеспечивающее аутентификацию с помощью USB-ключа.