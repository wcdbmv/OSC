\chapter{Аналитический раздел}

В первой главе выполняется постановка задачи на основе утверждённой руководителем КР темы и определяются требования и ограничения к разрабатываемому ПО.
Выполняется анализ поставленной задачи.
При этом анализируются методы или способы ее решения.
Проводится сравнительный анализ методов или способов решения и делается обоснованный выбор методов или способов решения поставленной задачи.
Например, при выборе задачи разработки драйвера операционной системы семейства Windows необходимо:
\begin{itemize}
	\item определить цель написания драйвера;
	\item на основе существующих классификаций WDM или WDF обоснованно выбрать тип драйвера;

	В случае выбора модели WDF необходимо проаналазировать и выбрать драйверную инфраструктуру: UMDF и KMDF и выбрать ту инфраструктуру, которая соответствует типу устройства, для которого решено написать драйвер.

	\item определить место выбранного драйвера в системе.

	В случае выбора модели WDF проанализировать особенности объектной модели WDF и иерархию объектов.

	В случае выбора драйвера WDM описать иерархию устройств и драйверов.
\end{itemize}

При выборе темы «Разработка драйверов для Unix и Unix-подобных систем» следует рассмотреть архитектуру подсистемы ввода/вывода, пространство имен, привести классификацию типов драйверов.
Обоснованно выбрать тип драйвера и определить, какой драйвер будет разрабатываться: встраиваемый или динамически загружаемый.
В процессе разработки может быть принято решение о необходимости модификации функций ядра, которое должно быть обосновано.

Если выбрана тема, связанная с модификацией исходного кода операционной системы (например, создание нового порта в ОС FreeBSD) необходимо проанализировать возможные
пути решения этой задачи и сделать обоснованный выбор.
Для патчей проанализировать особенности их разработки для решения конкретной задачи и перекомпиляции ядра.

\section*{Вывод}
