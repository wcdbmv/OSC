\chapter*{Введение}
\addcontentsline{toc}{chapter}{Введение}

\textit{Во введении обосновывается актуальность поставленной задачи, приводится краткий обзор существующих подходов и методов ее реализации, а также даются краткие характеристики
существующего программного обеспечения, полностью или частично реализующего заданные функции.}

В современном мире остро стоит вопрос обеспечения эффективной защиты информации.
В основном, у людей множество важной информации содержится в персональном компьютере, доступ к которому осуществляется, чаще всего, с помощью проверки правильности пароля.

Такой способ защиты информации не лишён недостатков: пароль к компьютеру можно узнать различными методами.
Более безопасным вариантом контроля доступа к компьютеру является многофакторная аутентификация, в частности, с дополнительной проверкой подключённого USB-ключа.

\textbf{Целью} данного курсового проекта является разработка загружаемого модуля ядра Linux, добавляющего в процесс загрузки операционной системы проверку наличия подключённого через интерфейс USB flash-накопителя с заданным серийным номером.
